\documentclass{article}
\usepackage[includeheadfoot,paperwidth=6in,paperheight=4.5in,
            margin=.5in]{geometry}
\usepackage{mathtools,amssymb,url}
\usepackage{lastpage,fancyhdr}
\pagestyle{fancy}
\fancyhf{}
\cfoot{\small \thepage{} / \pageref{LastPage}}

\newenvironment{slide}[1]{\fancyhead[C]{\Large \textsc{#1}}}{\pagebreak}
\def\labelitemi{$\gg$}

\title{\texttt{java.io} Tool Talk: Streams and Files}
\author{David Millard and Matthew Saltz \\
        \texttt{\{dmillard,saltzm\}@uga.edu}}
\date{January 14, 2013}

\begin{document}

\begin{slide}{}
    \maketitle
    \thispagestyle{empty}
\end{slide}

\setcounter{page}{1}

\begin{slide}{I/O Streams}
    \begin{itemize}
        \item An \emph{I/O Stream} provides an input source or output
            destination to a program (e.g. disk files, devices, other programs,
            memory arrays, etc.)
        \item Streams support many data types (including objects) and may be
            passive (forward data) or active (transform data)
        \item All streams provide the same model to the program: streams are
            sequences of data
    \end{itemize}
    \begin{figure}[h]
        \centering
        $\fbox{Data} \xrightarrow{\text{Input Stream}} \fbox{Program}$\\
        \vspace{.5em}
        $\fbox{Program} \xrightarrow{\text{Output Stream}} \fbox{Data}$
        \label{fig:streamdirections}
        \caption{Visualization of streams}
    \end{figure}
\end{slide}

\begin{slide}{Simple Example}
    To demonstrate common stream functions, we will use the very basic
    \emph{byte stream} with the following input file \texttt{xanadu.txt}, which
    contains:

    \begin{verbatim}
    In Xanadu did Kubla Khan
    A stately pleasure-dome decree:
    Where Alph, the sacred river, ran
    Through caverns measureless to man
    Down to a sunless sea.
    \end{verbatim}

    To follow along with the examples, run:
    
    \begin{verbatim}
    git clone https://github.com/dmillard/javaio_examples.git
    \end{verbatim}
\end{slide}

\begin{slide}{Simple Example (contd.)}
    \begin{itemize}
        \item Programs use \emph{byte streams} to directly input and output
            bytes.
        \item All byte stream classes are descended from \texttt{InputStream}
            and \texttt{OutputStream}
        \item This example uses \texttt{FileInputStream} and
            \texttt{FileOutputStream}
        \item The code for this example is in
            \texttt{Example1\_CopyBytes.java}
    \end{itemize}
\end{slide}

\begin{slide}{Notes on Example 1}
    \begin{itemize}
        \item Always close streams: this helps to prevent resource leaks
        \item As Example 1 showed, closing in a \texttt{finally} block helps
            ensure that all streams are appropriately closed
        \item Although Example 1 is very simple, it is too low level: as we are
            dealing with character data, we should be using \emph{character
            streams}
    \end{itemize}
\end{slide}

\begin{slide}{Character Streams}
    \begin{itemize}
        \item Character stream I/O is generally no more complex than byte
            stream I/O
        \item Character stream I/O, however, automatically translates from
            internal Unicode to the local character set (useful for
            internationalization)
        \item All character stream classes are descended from \texttt{Reader}
            and \texttt{Writer}.  As with byte streams, there are character
            stream classes specialized for files: \texttt{FileReader} and
            \texttt{FileWriter}
        \item The code for this example is in
            \texttt{Example2\_CopyCharacters.java}
    \end{itemize}
\end{slide}

\begin{slide}{Byte vs. Character Streams}
    \begin{itemize}
        \item Both examples have the same effect: what is different?
        \item Obviously, different classes: \texttt{FileReader} vs.
            \texttt{FileInputStream}
        \item Internally, \texttt{int c} holds different values:
            \begin{itemize}
                \item In the byte stream example, the last 8 bits hold the
                    byte to be copied
                \item In the character stream example, the last 16 bits hold
                    the character to be copied
            \end{itemize}
        \item Character streams can often be wrappers for byte streams:
            \texttt{FileReader} for example, uses \texttt{FileInputStream}
            internally
    \end{itemize}
\end{slide}

\begin{slide}{Line-Oriented I/O}
    \begin{itemize}
        \item Character I/O rarely happens character-by-character; commonly it
            is line-by-line
        \item A line is a string of characters with a line terminator
            (\texttt{"\textbackslash r\textbackslash n"},
             \texttt{"\textbackslash r"}, or \texttt{"\textbackslash n"})
         \item This example uses \texttt{BufferedReader} and
             \texttt{PrintWriter}, which will be discussed more in depth later
         \item The code for this example is in
             \texttt{Example3\_CopyLines.java}
     \end{itemize}
\end{slide}

\begin{slide}{Buffered Streams}
    \begin{itemize}
        \item Examples 1 and 2 use \emph{unbuffered} I/O, meaning that each
            read and write request is handled directly by the OS
        \item This is often slow: each request often will trigger disk access
        \item We can reduce this overhead with \emph{buffered streams}, which
            read data from a memory area (known as a \emph{buffer})
        \item Thus, OS calls are only made when the buffer is empty
        \item This functionality is implemented with
            \texttt{BufferedReader} and \texttt{BufferedWriter}, which are
            buffered drop-ins for \texttt{Reader} and \texttt{Writer}
        \item To buffer input to Example 2, we could have written:
    \end{itemize}
    \begin{verbatim}
    out = new BufferedWriter(new FileWriter("xanadu.txt"));
    \end{verbatim}
\end{slide}

\begin{slide}{Tokenizing and Translating Input}
    \begin{itemize}
        \item \texttt{Scanner} objects are useful for breaking down formatted
            input into individual tokens and translating according to type
        \item By default, \texttt{Scanner}s use whitespace characters to
            delimit tokens
        \item The code for this example is in \texttt{Example4\_Tokenize.java}
        \item Scanners can interpret character encoded data as types, as
            well \\ (e.g. "15.2" $\rightarrow 15.2$ and "1,234.5" $\rightarrow
            1234.5$)
        \item The code for this example is in \texttt{Example5\_Translate.java}
    \end{itemize}
\end{slide}

\begin{slide}{Data Streams}
    \begin{itemize}
        \item \emph{Data streams} support binary I/O of primitive data type
            values, as well as String values
        \item All data streams implement either the \texttt{DataInput}
            interface or the \texttt{DataOutput} interface
        \item The commonly used implementations are
            \texttt{DataInputStream}\\ and \texttt{DataOutputStream}
        \item The code for this example is in
            \texttt{Example6\_DataStreams.java}
        \item Of course, you should never use \texttt{double} to represent
            currency values, which leads us to...
    \end{itemize}
\end{slide}

\begin{slide}{Object Streams}
    \begin{itemize}
        \item Like data streams provide stream I/O for primitive data types,
            object streams support stream I/O for objects
        \item Most (not all) standard classes support serialization
        \item Classes supporting serialization implement \texttt{Serializable}
        \item The object streams are \texttt{ObjectInputStream} and
            \texttt{ObjectOutputStream}
        \item These implement \texttt{ObjectInput} and \texttt{ObjectOutput},
            which are sub-interfaces of \texttt{DataInput} and
            \texttt{DataOutput}
    \end{itemize}
\end{slide}

\begin{slide}{Complex Objects with Streams}
    \begin{itemize}
        \item \texttt{writeObject} and \texttt{readObject} are simple to use,
            but they contain some sophisticated object management logic
        \item Consider an object which contains references to other objects
        \item \texttt{writeObject(object0)} will write all objects necessary to
            reconstitute \texttt{object0}
        \item What if two objects written to the same stream both contain
            references to a single object?
            \begin{itemize}
                \item Both will refer to a single object when they are read
                    back
                \item A stream can only contain one copy of an object, buy any
                    number of references to that object
                \item Therefore, two writes of the same object is actually a
                    single write of the object and two writes of references to
                    the object
            \end{itemize}
        \item A simple example is in \texttt{Example7\_ObjectStream.java}
    \end{itemize}
\end{slide}

\begin{slide}{Files}
    \begin{itemize}
        \item Java \texttt{File}s are abstract representations of file and
            directory pathnames
        \item Java \texttt{File} provides some methods for manipulating
            pathnames, like \\\texttt{getParent()}, which gives the parent
            directory of the file
        \item Though all examples use pathnames directly as \texttt{FileReader}
            constructor parameters, they could use \texttt{File} objects with
            more flexibility
        \item For more control over how information is written to disk, we turn
            to \\\texttt{RandomAccessFile}s
    \end{itemize}
\end{slide}

\begin{slide}{Random Access Files}
    \begin{itemize}
        \item \texttt{RandomAccessFiles} support both reading and writing to a
            random access file
        \item A random access file behaves like a large array of bytes on disk
        \item There is an index, or cursor, called the \emph{file pointer},
            which which determines at which point bytes are written and read
        \item The \emph{file pointer} can be read by the
            \texttt{getFilePointer} method and set by the \texttt{seek} method
        \item \texttt{RandomAccessFile} implements \texttt{DataOutput} and
            \texttt{DataInput}, and therefore provides the reading and writing
            functionality we would expect from streams
    \end{itemize}
\end{slide}

\begin{slide}{References}
    \begin{itemize}
        \item
            \url{docs.oracle.com/javase/7/docs/api/java/io/RandomAccessFile.html}
        \item
            \url{docs.oracle.com/javase/7/docs/api/java/io/File.html}
        \item
            \url{docs.oracle.com/javase/tutorial/essential/io/index.html}
    \end{itemize}
\end{slide}

\end{document}
